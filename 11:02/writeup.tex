\NeedsTeXFormat{LaTeX2e}
\documentclass[11pt]{article}
\usepackage{url}
\usepackage{amsmath}
\usepackage{amsthm}
\usepackage{amssymb}
\usepackage{mathpartir}
\usepackage{graphicx}
\usepackage{comment}


%commands for PL

\newcommand{\deftech}[1]{\textbf{#1}}
\newcommand\mrel{\mathop{\mathbf{r}}}
\newcommand\morel{\mathop{\mathbf{r}'}}
\newcommand\menv{\rho}
\newcommand\mval{v}
\newcommand\mans{a}
\newcommand\mint{i}
\newcommand\moint{j}
\newcommand\mbool{b}

\newcommand\plug[2]{#1[#2]}

\newcommand\merr{r}

\newcommand\mctx{\mathcal{C}}
\newcommand\mectx{\mathcal{E}}
\newtheorem{theorem}{Theorem}
\newtheorem{lemma}{Lemma}
\newtheorem{definition}{Definition}
\newcommand\Err{\mathit{Err}}
\newcommand\Plus{\mathit{Plus}}
\newcommand\Mult{\mathit{Mult}}
\newcommand\Succ{\mathit{Succ}}
\newcommand\Pred{\mathit{Pred}}
\newcommand\Eq{\mathit{Eq}}
\newcommand\True{\mathit{True}}
\newcommand\False{\mathit{False}}
\newcommand\If{\mathit{If}}
\newcommand\Div{\mathit{Div}}

\newcommand\reduce{\mathop{\mathbf{b}}}

\newcommand\areducename{\mathbf{a}}
\newcommand\areduce[2]{#1\;\areducename\;#2}

\newcommand\step{\rightarrow_\mathbf{b}}
\newcommand\multistep{\rightarrow^\star_\mathbf{b}}

\newcommand\astdstep{\longmapsto_{\reduce}}
\newcommand\astdmultistep{\longmapsto^\star_{\reduce}}

\newcommand\breducename{\mathbf{b}}
\newcommand\errreducename{\mathbf{err}}
\newcommand\propreducename{\mathbf{prop}}
\newcommand\bvreducename{\mathbf{bv}}
\newcommand\bstepname{\rightarrow_{\breducename}}
\newcommand\bmultistepname{\rightarrow^\star_{\breducename}}

\newcommand\bmultistep[3]{#1\vdash #2\;\bmultistepname\;#3}
\newcommand\bstdstep[3]{#1\vdash #2\;{\longmapsto_{\breducename}}\;#3}
\newcommand\bstdmultistep[3]{#1\vdash #2\;{\longmapsto^\star_{\breducename}}\;#3}

\newcommand\breduce[3]{#1 \vdash {#2}\;\breducename\; {#3}}
\newcommand\errreduce[3]{#1 \vdash {#2}\;\errreducename\; {#3}}
\newcommand\propreduce[3]{#1 \vdash {#2}\;\propreducename\; {#3}}
\newcommand\bvreduce[3]{#1 \vdash {#2}\;\bvreducename\; {#3}}
\newcommand\bstep[3]{#1 \vdash {#2}\;\rightarrow_{\breducename}\; {#3}}
\newcommand\bclosedstep[2]{{#1}\;\rightarrow_{\breducename}\; {#2}}

\newcommand\laxparstep{\rightrightarrows_\mathbf{a}}
\newcommand\maxparstep{\rightrightarrows'_\mathbf{a}}


\newcommand\Arith{\mathcal{A}}
\newcommand\Barith{\mathcal{B}}

\newcommand\Var{\mathit{Var}}

\newcommand{\mvar}{x}
\newcommand\s[1]{\mathit{#1}}

%commands for CW-Privacy

\newcommand{\xdown}[1]{X_{#1 \downarrow}}
\newcommand{\xup}[1]{X_{#1 \uparrow}}
\newcommand{\xdownk}[2]{X_{#1 \downarrow #2}}
\newcommand{\xupk}[2]{X_{#1 \uparrow #2}}
\newcommand{\xblock}[1]{[x_{#1 1}, x_{#1 2} , \dots, x_{#1 \frac{1}{c}}]^{\top}}
\newcommand{\conpr}[2]{Pr[#1\,|\,#2]}
\newcommand{\priv}{{\bf priv}(X)}
\newcommand{\alt}[1]{{\bf alt}(X_{#1})}
\newcommand{\xbot}[1]{x_{#1 \frac{1}{c}}}
\newcommand{\cwp}[1]{(\epsilon, \delta, \Delta_{#1}, \Gamma)}
\newcommand{\xmid}[1]{X_{#1 \updownarrow}}
\newcommand{\xm}[1]{x_{#1 \circ}}

%commands for SN- Controal
\newcommand{\cf}[2]{f(t_{#2}\,|\, t_{#1} ; \alpha_{#1,#2})}
\newcommand{\cg}[2]{g(t_{#2}\,|\, t_{#1} ; \alpha_{#1,#2})}



\title{Information diffusion as a control model}
\date{\today}
\begin{document}
\maketitle
\begin{enumerate}
\item Consider the discrete time step. Let $X_{t} = [p_{1}^{(t)}, \dots, p_{n}^{(t)}]^{\top}$, where each $p_{i}^{(t)}$ denote the probability that node $i$ has been activated by time step $t$.
\item For all pair of nodes $i,j$, $\alpha_{j,i}$ is a parameter modeling how frequent information spread from $j$ to $i$. $\alpha_{j,i}=0$ means no link.  
\item Let $f(t_{i}|t_{j}; \alpha_{j,i})$ be the conditional likelihood of transmission from $j$ to $i$. e.g., $f$ can be the pdf of exp, pow, ray distributions.
\item $p_{i}^{(t+1)} = p_{1}^{(t)} + (1-p_{1}^{(t)})\delta_{i}^{t+1} $, where $\delta_{i}^{t+1}$ denote the probability that node $i$ is activated during $t$ to $t+1$.
\item Let $F_{i}^{(t+1)} (t_{j}; \alpha_{j,i}) = \int_t^{t+1} \! f(t_{i}|t_{j}; \alpha_{j,i}) \, \mathrm{d}t_{i}$. Notice these $F_{i}^{(t+1)} (t_{j}; \alpha_{j,i})$ can be precomputed.
Let $F_{i}^{(t+1)}(j;\alpha_{j,i}) = \Sigma_{t_j=1}^{t} F_{i}^{(t+1)} (t_{j}; \alpha_{j,i}) \cdot Pr(\text{j is activated at $t_{j}$})$. Then, $\delta_{i}^{(t+1)} = 1- \Pi_{j\neq i} (1- F_{i}^{(t+1)}(j;\alpha_{j,i}))$.
\item $Pr(\text{j is activated at $t_{j}$})$ can be calculated from $X_{1}, \dots, X_{t}$. Specifically, $Pr(\text{j is activated at $t_{j}$}) = p_{j}^{t_{j}}\Pi_{t<t_{j}} (1-p_{j}^{t})$.
\item Control inputs can be (1) observing some nodes be activated; (2) modifying $\alpha_{j,i}$.
\item Evaluation
\end{enumerate}

\end{document}
